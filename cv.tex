%-----------------------------------------------------------------------------------------------------------------------------------------------%
%	The MIT License (MIT)
%
%   Copyright (c) 2021 Mike Lubibets
%	Copyright (c) 2021 Philip Empl (original template authorship)
%
%	Permission is hereby granted, free of charge, to any person obtaining a copy
%	of this software and associated documentation files (the "Software"), to deal
%	in the Software without restriction, including without limitation the rights
%	to use, copy, modify, merge, publish, distribute, sublicense, and/or sell
%	copies of the Software, and to permit persons to whom the Software is
%	furnished to do so, subject to the following conditions:
%	
%	THE SOFTWARE IS PROVIDED "AS IS", WITHOUT WARRANTY OF ANY KIND, EXPRESS OR
%	IMPLIED, INCLUDING BUT NOT LIMITED TO THE WARRANTIES OF MERCHANTABILITY,
%	FITNESS FOR A PARTICULAR PURPOSE AND NONINFRINGEMENT. IN NO EVENT SHALL THE
%	AUTHORS OR COPYRIGHT HOLDERS BE LIABLE FOR ANY CLAIM, DAMAGES OR OTHER
%	LIABILITY, WHETHER IN AN ACTION OF CONTRACT, TORT OR OTHERWISE, ARISING FROM,
%	OUT OF OR IN CONNECTION WITH THE SOFTWARE OR THE USE OR OTHER DEALINGS IN
%	THE SOFTWARE.
%	
%
%-----------------------------------------------------------------------------------------------------------------------------------------------%


%============================================================================%
%
%	DOCUMENT DEFINITION
%
%============================================================================%

\documentclass[10pt,A4,english]{article}	


%----------------------------------------------------------------------------------------
%	ENCODING
%----------------------------------------------------------------------------------------

% we use utf8 since we want to build from any machine
\usepackage[utf8]{inputenc}		
\usepackage[USenglish]{isodate}
\usepackage{fancyhdr}
\usepackage[numbers]{natbib}

%----------------------------------------------------------------------------------------
%	LOGIC
%----------------------------------------------------------------------------------------

% provides \isempty test
\usepackage{xstring, xifthen}
\usepackage{enumitem}
\usepackage[english]{babel}
\usepackage{blindtext}
\usepackage{pdfpages}
\usepackage{changepage}
%----------------------------------------------------------------------------------------
%	FONT BASICS
%----------------------------------------------------------------------------------------

% some tex-live fonts - choose your own

%\usepackage[defaultsans]{droidsans}
%\usepackage[default]{comfortaa}
%\usepackage{cmbright}
\usepackage[default]{raleway}
%\usepackage{fetamont}
%\usepackage[default]{gillius}
%\usepackage[light,math]{iwona}
%\usepackage[thin]{roboto} 

% set font default
\renewcommand*\familydefault{\sfdefault} 	
\usepackage[T1]{fontenc}

% more font size definitions
\usepackage{moresize}

%----------------------------------------------------------------------------------------
%	FONT AWESOME ICONS
%---------------------------------------------------------------------------------------- 

% include the fontawesome icon set
\usepackage{fontawesome}

% use to vertically center content
% credits to: http://tex.stackexchange.com/questions/7219/how-to-vertically-center-two-images-next-to-each-other
\newcommand{\vcenteredinclude}[1]{\begingroup
\setbox0=\hbox{\includegraphics{#1}}%
\parbox{\wd0}{\box0}\endgroup}
\newcommand{\tab}[1]{\hspace{.2\textwidth}\rlap{#1}}
% use to vertically center content
% credits to: http://tex.stackexchange.com/questions/7219/how-to-vertically-center-two-images-next-to-each-other
\newcommand*{\vcenteredhbox}[1]{\begingroup
\setbox0=\hbox{#1}\parbox{\wd0}{\box0}\endgroup}

% icon shortcut
\newcommand{\icon}[3] { 							
	\makebox(#2, #2){\textcolor{maincol}{\csname fa#1\endcsname}}
}	


% icon with text shortcut
\newcommand{\icontext}[4]{ 						
	\vcenteredhbox{\icon{#1}{#2}{#3}}  \hspace{2pt}  \parbox{0.9\mpwidth}{\textcolor{#4}{#3}}
}

% icon with website url
\newcommand{\iconhref}[5]{ 						
    \vcenteredhbox{\icon{#1}{#2}{#5}}  \hspace{2pt} \href{#4}{\textcolor{#5}{#3}}
}

% icon with email link
\newcommand{\iconemail}[5]{ 						
    \vcenteredhbox{\icon{#1}{#2}{#5}}  \hspace{2pt} \href{mailto:#4}{\textcolor{#5}{#3}}
}

%----------------------------------------------------------------------------------------
%	PAGE LAYOUT  DEFINITIONS
%----------------------------------------------------------------------------------------

% page outer frames (debug-only)
% \usepackage{showframe}		

% we use paracol to display breakable two columns
\usepackage{paracol}
\usepackage{tikzpagenodes}
\usetikzlibrary{calc}
\usepackage{lmodern}
\usepackage{multicol}
\usepackage{lipsum}
\usepackage{atbegshi}
% define page styles using geometry
\usepackage[a4paper]{geometry}

% remove all possible margins
\geometry{top=1cm, bottom=1cm, left=1cm, right=1cm}

\usepackage{fancyhdr}
\pagestyle{empty}

% space between header and content
% \setlength{\headheight}{0pt}

% indentation is zero
\setlength{\parindent}{0mm}

%----------------------------------------------------------------------------------------
%	TABLE /ARRAY DEFINITIONS
%---------------------------------------------------------------------------------------- 

% extended aligning of tabular cells
\usepackage{array}

% custom column right-align with fixed width
% use like p{size} but via x{size}
\newcolumntype{x}[1]{%
>{\raggedleft\hspace{0pt}}p{#1}}%


%----------------------------------------------------------------------------------------
%	GRAPHICS DEFINITIONS
%---------------------------------------------------------------------------------------- 

%for header image
\usepackage{graphicx}

% use this for floating figures
% \usepackage{wrapfig}
% \usepackage{float}
% \floatstyle{boxed} 
% \restylefloat{figure}

%for drawing graphics		
\usepackage{tikz}			
\usepackage{ragged2e}	
\usetikzlibrary{shapes, backgrounds,mindmap, trees}

%----------------------------------------------------------------------------------------
%	Color DEFINITIONS
%---------------------------------------------------------------------------------------- 
\usepackage{transparent}
\usepackage{color}

% primary color
\definecolor{maincol}{RGB}{ 64,64,64}

% accent color, secondary
% \definecolor{accentcol}{RGB}{ 250, 150, 10 }

% dark color
\definecolor{darkcol}{RGB}{ 70, 70, 70 }

% light color
\definecolor{lightcol}{RGB}{245,245,245}

\definecolor{accentcol}{RGB}{59,77,97}



% Package for links, must be the last package used
\usepackage[hidelinks]{hyperref}

% returns minipage width minus two times \fboxsep
% to keep padding included in width calculations
% can also be used for other boxes / environments
\newcommand{\mpwidth}{\linewidth-\fboxsep-\fboxsep}
	


%============================================================================%
%
%	CV COMMANDS
%
%============================================================================%

%----------------------------------------------------------------------------------------
%	 CV LIST
%----------------------------------------------------------------------------------------

% renders a standard latex list but abstracts away the environment definition (begin/end)
\newcommand{\cvlist}[1] {
	\begin{itemize}{#1}\end{itemize}
}

%----------------------------------------------------------------------------------------
%	 CV TEXT
%----------------------------------------------------------------------------------------

% base class to wrap any text based stuff here. Renders like a paragraph.
% Allows complex commands to be passed, too.
% param 1: *any
\newcommand{\cvtext}[1] {
	\begin{tabular*}{1\mpwidth}{p{0.98\mpwidth}}
		\parbox{1\mpwidth}{#1}
	\end{tabular*}
}
\newcommand{\cvtextsmall}[1] {
	\begin{tabular*}{0.8\mpwidth}{p{0.8\mpwidth}}
		\parbox{0.8\mpwidth}{#1}
	\end{tabular*}
}
%----------------------------------------------------------------------------------------
%	CV SECTION
%----------------------------------------------------------------------------------------

% Renders a a CV section headline with a nice underline in main color.
% param 1: section title
\newcommand{\cvsection}[1] {
	\vspace{14pt}
	\cvtext{
		\textbf{\LARGE{\textcolor{darkcol}{#1}}}\\[-4pt]
		\textcolor{accentcol}{ \rule{0.2\textwidth}{1.5pt} } \\
	}
}

\newcommand{\cvsectionsmall}[1] {
	\vspace{14pt}
	\cvtext{
		\textbf{\Large{\textcolor{darkcol}{#1}}}\\[-4pt]
		\textcolor{accentcol}{ \rule{0.2\textwidth}{1.5pt} } \\
	}
}

\newcommand{\cvheadline}[1] {
	\vspace{16pt}
	\cvtext{
		\textbf{\Huge{\textcolor{accentcol}{#1}}}\\[-4pt]
		 
	}
}

\newcommand{\cvsubheadline}[1] {
	\vspace{16pt}
	\cvtext{
		\textbf{\huge{\textcolor{darkcol}{#1}}}\\[-4pt]
		 
	}
}
%----------------------------------------------------------------------------------------
%	META SKILL
%----------------------------------------------------------------------------------------

% Renders a progress-bar to indicate a certain skill in percent.
% param 1: name of the skill / tech / etc.
% param 2: level (for example in years)
% param 3: percent, values range from 0 to 1
\newcommand{\cvskill}[3] {
	\begin{tabular*}{1\mpwidth}{p{0.72\mpwidth}  r}
 		\textcolor{black}{\textbf{#1}} & \textcolor{maincol}{#2}\\
	\end{tabular*}%
	
	\hspace{4pt}
	\begin{tikzpicture}[scale=1,rounded corners=2pt,very thin]
		\fill [lightcol] (0,0) rectangle (1\mpwidth, 0.15);
		\fill [accentcol] (0,0) rectangle (#3\mpwidth, 0.15);
  	\end{tikzpicture}%
}


%----------------------------------------------------------------------------------------
%	 CV EVENT
%----------------------------------------------------------------------------------------

% Renders a table and a paragraph (cvtext) wrapped in a parbox (to ensure minimum content
% is glued together when a pagebreak appears).
% Additional Information can be passed in text or list form (or other environments).
% the work you did
% param 1: time-frame i.e. Sep 14 - Jan 15 etc.
% param 2:	 event name (job position etc.)
% param 3: Customer, Employer, Industry
% param 4: Short description
% param 5: work done (optional)
% param 6: technologies include (optional)
% param 7: achievements (optional)
\newcommand{\cvevent}[7] {
	
	% we wrap this part in a parbox, so title and description are not separated on a pagebreak
	% if you need more control on page breaks, remove the parbox
	\parbox{\mpwidth}{
		\begin{tabular*}{1\mpwidth}{p{0.66\mpwidth}  r}
	 		\textcolor{black}{\textbf{#2}} & \colorbox{accentcol}{\makebox[0.32\mpwidth]{\textcolor{white}{\textbf{#1}}}} \\
			\textcolor{maincol}{#3} & \\
		\end{tabular*}\\[8pt]
	
		\ifthenelse{\isempty{#4}}{}{
			\cvtext{#4}\\
		}
	}
	\vspace{14pt}
}


%----------------------------------------------------------------------------------------
%	 CV META EVENT
%----------------------------------------------------------------------------------------

% Renders a CV event on the sidebar
% param 1: title
% param 2: subtitle (optional)
% param 3: customer, employer, etc,. (optional)
% param 4: info text (optional)
\newcommand{\cvmetaevent}[4] {
	\textcolor{maincol} { \cvtext{\textbf{\begin{flushleft}#1\end{flushleft}}}}

	\ifthenelse{\isempty{#2}}{}{
	\textcolor{black} {\cvtext{\textbf{#2}} }
	}

	\ifthenelse{\isempty{#3}}{}{
		\cvtext{{ \textcolor{maincol} {#3} }}\\
	}

	\cvtext{#4}\\[14pt]
}

%---------------------------------------------------------------------------------------
%	QR CODE
%----------------------------------------------------------------------------------------

% Renders a qrcode image (centered, relative to the parentwidth)
% param 1: percent width, from 0 to 1
\newcommand{\cvqrcode}[1] {
	\begin{center}
		\includegraphics[width={#1}\mpwidth]{qrcode}
	\end{center}
}


% HEADER AND FOOOTER 
%====================================
\newcommand\Header[1]{%
\begin{tikzpicture}[remember picture,overlay]
\fill[accentcol]
  (current page.north west) -- (current page.north east) --
  ([yshift=50pt]current page.north east|-current page text area.north east) --
  ([yshift=50pt,xshift=-3cm]current page.north|-current page text area.north) --
  ([yshift=10pt,xshift=-5cm]current page.north|-current page text area.north) --
  ([yshift=10pt]current page.north west|-current page text area.north west) -- cycle;
\node[font=\sffamily\bfseries\color{white},anchor=west,
  xshift=0.7cm,yshift=-0.32cm] at (current page.north west)
  {\fontsize{12}{12}\selectfont {#1}};
\end{tikzpicture}%
}

\newcommand\Footer[1]{%
\begin{tikzpicture}[remember picture,overlay]
\fill[lightcol]
  (current page.south east) -- (current page.south west) --
  ([yshift=-80pt]current page.south east|-current page text area.south east) --
  ([yshift=-80pt,xshift=-6cm]current page.south|-current page text area.south) --
  ([xshift=-2.5cm,yshift=-10pt]current page.south|-current page text area.south) --	
  ([yshift=-10pt]current page.south east|-current page text area.south east) -- cycle;
\node[yshift=0.32cm,xshift=9cm] at (current page.south) {\fontsize{10}{10}\selectfont \textbf{\thepage}};
\end{tikzpicture}%
}


%=====================================
%============================================================================%
%
%
%
%	DOCUMENT CONTENT
%
%
%
%============================================================================%
\begin{document}

\columnratio{0.31}
\setlength{\columnsep}{2.2em}
\setlength{\columnseprule}{4pt}
\colseprulecolor{white}


% LEBENSLAUF HIERE
\AtBeginShipoutFirst{\Header{Mike Lubinets}\Footer{1}}
\AtBeginShipout{\AtBeginShipoutAddToBox{\Header{Mike Lubinets}\Footer{2}}}

\newpage

\colseprulecolor{lightcol}
\columnratio{0.31}
\setlength{\columnsep}{2.2em}
\setlength{\columnseprule}{4pt}
\begin{paracol}{2}


\begin{leftcolumn}
%---------------------------------------------------------------------------------------
%	META IMAGE
%----------------------------------------------------------------------------------------
\includegraphics[width=\linewidth]{resources/photo.jpg}	%trimming relative to image size

%---------------------------------------------------------------------------------------
%	META SKILLS
%----------------------------------------------------------------------------------------
	\fcolorbox{white}{white}{\begin{minipage}[c][1.5cm][c]{1\mpwidth}
		\LARGE{\textbf{\textcolor{maincol}{Mike Lubinets}}} \\[2pt]
		\normalsize{ \textcolor{maincol} {Senior Rust Developer} }
\end{minipage}} \\

\icontext{CaretRight}{12}{Living in UAE | Abu Dhabi}{black}\\[6pt]
\icontext{CaretRight}{12}{Open to relocation}{black}\\[6pt]

\cvsection{Skills}

\cvskill{GNU/Linux} {10+ yrs.} {0.8} \\[-2pt]

\cvskill{Rust} {5+ yrs.} {0.9} \\[-2pt]

\cvskill{Research \& Development} {4+ yrs.} {0.64} \\[-2pt]

\cvskill{Computer Networks} {2+ yrs.} {0.6} \\[-2pt]

\cvskill{Distributed legder \newline technology} {3+ yrs.} {0.5} \\[-2pt]

\cvskill{Web development} {3+ yrs.} {0.4} \\[-2pt]

\cvskill{Embedded Systems} {2+ yrs.} {0.3} \\[-2pt]

\cvskill{English} {C1} {0.8} \\[-2pt]

\newpage
%---------------------------------------------------------------------------------------
%	EDUCATION
%----------------------------------------------------------------------------------------
\cvsection{Education}

\cvmetaevent
{2016 - 2018}
{Computer Science (B.Sc.)}
{Innopolis University}
{
	GPA: \textit{3.44/4} \newline 
	Bachelor's thesis: \glqq Modular Language Server for SLang Programming Language\grqq.
}

\cvmetaevent
{2014 - 2016}
{Computer Science (B.Sc.)}
{Saint Petersburg State University of Aerospace and Instrumentation}

%
\cvsection{Projects}
\cvlist{
	\item \hyperlink{https://github.com/mersinvald/aquamarine}{\textbf{aquamarine}}\\ Inline diagrams for rustdoc, used by Google
	\item \hyperlink{https://github.com/mersinvald/reed-solomon-rs}{\textbf{reed-solomon-rs}}\\ Reed-Solomon BCH implementation in Rust
	\item \hyperlink{https://github.com/mersinvald/batch_resolve}{\textbf{batch-resolver}} \\ Batch asynchronous DNS resolver
	\item \hyperlink{https://github.com/mersinvald/disciplinator}{\textbf{disciplinator}}\\ Hobby project to break sedentary lifestyle using FitBit API
	\item \hyperlink{https://crates.io/users/mersinvald}{\textbf{crates.io/users/mersinvald}}\\ Other small libraries and tools contributed to Rust ecosystem
}

\cvsection{Interests}

\icontext{CaretRight}{12}{Hiking}{black}\\[6pt]
\icontext{CaretRight}{12}{DIY}{black}\\[6pt]
\icontext{CaretRight}{12}{Programmable Keyboards}{black}\\[6pt]
\icontext{CaretRight}{12}{Cooking}{black}\\[6pt]
\icontext{CaretRight}{12}{Winter Sports}{black}\\[6pt]

\cvsection{Contact}

\icontext{MobilePhone}{16}{+49 176 27 95 28 13}{black}\\[6pt]
\iconemail{Envelope}{16}{me@mkl.dev}{me@mkl.dev}{black}\\[6pt]
\iconhref{Github}{16}{github.com/mersinvald}{https://www.github.com/mersinvald}{black}\\[6pt]
	
%\cvqrcode{0.3}

\end{leftcolumn}
\begin{rightcolumn}
%---------------------------------------------------------------------------------------
%	TITLE  HEADER
%----------------------------------------------------------------------------------------


%---------------------------------------------------------------------------------------
%	PROFILE
%----------------------------------------------------------------------------------------
\cvsection{About}
\vspace{4pt}

\cvtext{
    I am a Rust enthusiast who enjoys taking part in creation of new products 
    and making them reliable and loved. \\
    Currently I'm looking for on-site opportunities in UAE or relocation options in other countries,
	which would allow me to grow in the systems development, networks, software design and security while I'm preparing for my MSc. \\
	I'm looking forward to working with awesome people and I'll be doing my best to ensure you have the cleanest type-driven APIs 
	in the software that's reliable and performant. \\
    If this profile fits you, you are very welcome to reach out.
}


%---------------------------------------------------------------------------------------
%	WORK EXPERIENCE
%----------------------------------------------------------------------------------------

\vspace{10pt}
\cvsection{Work experience}
\vspace{4pt}


\cvevent
{Sep/2021 - today}
	{Senior Security Engineer}
	{Technology Innovation Instutute\newline Digital Science Research Centre}
	{
		Being an engineer in a research institution, my responsibilities spanned across variery of fields and support jobs for the research personnel.
		This included tooling development, infrastructure design and maintenance,
		security research of both software and hardware (Software Engineering, DevOps, SRE).
		\vfill\null
		During my occupation, I have
		\begin{itemize}
			\item developed various tools, e.g a triager for vulnerability classification to aid fuzzing.
			\item designed and implemented an infrastructure architecture of a distributed fuzzing platform.
			\item implemented initial stages of security research with a number of hardware targets, to provide research personnel with insigt data.
			\item setup and maintained department IT services (DevOps/SRE).
			\item setup and maintained physical infrastructure, virtualization, and networks.
			\item prepared and hosted a number of internal technical demo events.
			\item hosted a technical workshop at BlackHat conference in Las Vegas.
			\item lead hardware aquizition process for the department needs.
		\end{itemize}
		\textit{Technical Stack}: Rust, C++, Python, CLang Sanitizers, AFL++, QEMU/KVM, Kubernetes
	}
	\vfill\null

\cvevent
{Oct/2019 - Jul/2021}
	{Senior Rust Developer}
	{peaq GmbH}
    {
		My main occupation at peaq was the production of the peaq DLT platform and the related
		products. I have been working on the smart contracts execution runtime, blockchain storage, 
		and the implementation of the business requirements into the smart contracts running on the said platform. 
		\vfill\null
		During my time at peaq, I have
		\begin{itemize}
			\setlength\itemsep{-0.1em}
			\item successfully delivered the IoT and business logic parts of the technical PoC demo for the big partner in German automotive industry.
			\item designed and implemented the core layer of the blockchain-backed role based access control system, which later became the company's flagship product.
			\item lead the integration of the said RBAC system at the NTT datacenter, in partnership with FATH Mechatronics.
			\item helped two junior emloyees in their onboarding and integration into the project.
			\item been a part of the great hard-working team, which didn't miss a single tight deadline (and there were many!).
		\end{itemize}
		\textit{Technical Stack}: Rust, C++, CLang Sanitizers, Exonum, RocksDB, GitLab CI, Docker
	}
	\vfill\null

\cvevent
{Jul/2018 - May/2019}
	{Rust Developer}
	{Ethereum Classic Labs\newline EVM/Compiler Team\newline Tooling Team}
	{
		I was in charge of development and maintenance of the SputnikVM Ethereum Virtual Machine,
		as well as its integration into several Ethereum Clients, such as \textit{go-ethereum} and \textit{multi-geth}.
		\vfill\null
		Later, I switched to the tooling team, and worked on the Rust instruments around the OpenRPC protocol definition standard.
		\vfill\null
		During my time at ETCLabs, I have
		\begin{itemize}
			\setlength\itemsep{-0.1em}
			\item released SputnikVM 0.11, that allowed integration of the VM into the cross-chain clients that support multiple Ethereum-based networks.
			\item streamlined the EVM testing framework used by SputnikVM, which allowed us to use whole Ethereum VM test suite.
			\item implemented support of Spurious Dragon, Byzantium and Constantinople sets of EIPs.
			\item maintained a number of other products implemented in Rust.
		\end{itemize}
		\textit{Technical Stack}: Rust, C, Go, JsonRPC, Ethereum, Jenkins CI, Docker, Vagrant
	}
	\vfill\null	
	\vfill\null
	
\cvevent
{Oct/2017 - Jul/2018}
	{Freelance Rust Developer}
	{Zamar AG}
	{
		As a freelance Rust developer I implemented a number of services for the client Airline Baggage processing system, 
		including IATA/BagMessage protocol implementation in Rust.
		\vfill\null
		Each module has been designed to be panic-free, highly reliable and performant.
		\vfill\null
		\textit{Technical Stack}: Async Rust, Tokio, SOAP, PostgreSQL
	}
	\vfill\null
	
\cvevent
    {Mar/2017 - Jul/2018}
	{Systems C Developer}
	{AnP Workcell\newline Systems Software Team}
	{
		In the Systems Software team, I have 
		\begin{itemize}
			\setlength\itemsep{-0.1em}
			\item ported low-level parts of \textit{uclibc-ng} to the \textit{Elbrus e2k} processor architecture.
			\item established a continuous integration infrastructure for running automation tests of the E2K builds of the Linux Kernel and libc.\\
				  The integration of the generated test reports with Gitlab Issues have helped our team tremendously with keeping track of
			      thousands of regressions.
		\end{itemize}
		\textit{Technical Stack}: C, GDB, Valgring, GNU/Linux, Gitlab CI
	}
	\vfill\null

\cvevent
	{Nov/2015 - Jul/2016}
	{Trainee Embedded C++ Developer}
	{GK Scout\newline New Products R\&D Team}
	{
		On my first job, I got the experience intoductory to the embedded development with Cortex-M, 
		and to the product software development in general.
		\vfill\null
		During my internship, I have
		\begin{itemize}
			\setlength\itemsep{-0.1em}
			\item implemented a NAND-flash driver with wear-levelling and Reed-Solomon BCH based error correction.
			\item improved speed and reliability of the proprietary networking protocol implementation which allowed transport of 3x bigger packets on the same MCU.
			\item unexpectedly become an only developer on a prototype firmware project and finished it before the scheduled release date.
		\end{itemize}
		\textit{Technical Stack}: Embedded C, C++, ARM Cortex-M, GDB, JTAG
	}
	\vfill\null

\end{rightcolumn}
\end{paracol}
%---------------------------------------------------------------------------------------
%	Recommendations
%----------------------------------------------------------------------------------------
\pagebreak
\cvsection{Recommendations}
\vspace{4pt}

\cvevent
	{August 2019, Senior Thesis Advisor}
	{Eugene Zouev}
	{Professor — Innopolis University\newline \hyperlink{mailto://eugene.zueff@gmail.com}{\textit{eugene.zueff@gmail.com}}}
	{
		It is my pleasure to write this letter of recommendation for Mikhail Lubinets, whom I have known for 3 years as a student of Innopolis University. I was a supervisor for Mikhail’s undergraduate research and taught several classes that he took as a part of his educational program.
		\vfill\null
		As a student, Mikhail always displayed a consistently high level in all of my courses and his grades are a good indicator of his abilities. However, it doesn’t end there, as a truly investigative person, Mikhail always went an extra mile to extend his knowledge and skills, when interested in some topic. Having observed Mikhail in several courses (“Introduction to Programming II” and “Functional Programming and Scala”), that involve collaboration with other students, I can tell that he makes a good team player: responsible and always ready to help.
		\vfill\null
		Mikhail has graduated from Innopolis with a BS thesis project “Architecture of Modular Language Server for the SLang programming language” under my supervising. During his undergraduate research, Mikhail demonstrated strong work ethic, professionalism, and high motivation. He is quite disciplined, organized person; he has never had any trouble with meeting deadlines for his assignments.
		\vfill\null
		On a personal level, Mikhail can be characterized as a very sympathetic and communicative person.
		To sum up, I strongly recommend Mikhail as a candidate for a research or work position without any reservations.
		\vfill\null
		Should you have any more questions feel free to contact me for further information.
	}
	\vfill\null

\cvevent
	{August 2019, Fellow Student}
	{Alena Iureva}
	{MSc in Data Engineering — Jacobs University Bremen\newline \hyperlink{mailto://aiureva@jacobs-university.de}{\textit{aiureva@jacobs-university.de}}}
	{
		Leading a student community is a busy and challenging thing to do, and I am happy I had Mike here nearby, with him taking care of study groups on courses he was best at, and helping out at admin stuff like planning and supplies.
		\vfill\null
		He is, as well, a true enthusiast and evangelist of Rust, who eventually got me to try the language and start contributing to open-source time to time, which led me to overall wider communication with international IT community.
		\vfill\null
		Mike has been leading by example, and a lot, being highly competitive and pushing us all to test our limits in abilities to accomplish better results, faster, showing example of impeccable confidence and I-can-do-it attitude. At the same time, he is always ready to help should someone be asking or seeking answers, taking his time to explain, review or give feedback.
		\vfill\null
		Overall, I see him as someone who really enjoys his work and wants to acquire as much expertise at it as possible, and I still keep hearing about projects he starts in his free time, or exciting new technologies he’s currently learning.
	}
	\vfill\null

\cvevent
	{March 2019, Direct Manager}
	{Darcy Reno}
	{Managing Director at ETCDEV Team\newline \hyperlink{https://linkedin.com/in/darcyreno}{\textit{linkedin/darcyreno}}}
	{
		Mike is a pleasure to work with - He has an excellent work ethic and is always looking for ways to make improvement in the product, the process and himself. I would gladly work with him in the future!
	}
	\vfill\null


% hofixes to create fake-space to ensure the whole height is used
\mbox{}
\vfill
\mbox{}
\vfill
\mbox{}
\vfill
\mbox{}
\vfill
\mbox{}
\vfill
\mbox{}
\vfill
\mbox{}


\end{document}